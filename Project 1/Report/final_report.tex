\documentclass{article}
\usepackage[utf8]{inputenc}
\usepackage{amsmath}
\usepackage{tabu}

\title{IT3105: Project 1 \\ Implementation and Comparison of AI Algorithms to Solve The n-Queens Problem}
\author{Aksel Reiten and Magnus Moan}
\date{September 2016}

\begin{document}

\maketitle
\tableofcontents

% Part 1
\section{Part 1}
\subsection{Board representation}
We chose to represent our board in a one-dimensional array of size $n$, where each index represents a column and each number represents a row in the board. We chose this representation because it would (1) allow us to store and manipulate data in an efficient way with respect to time and space constraints and (2) easily guarantee absence of conflicts in both rows and columns.

\subsection{Heuristic function}
We are doing two different cut-offs in the search space to make the backtracking search go faster. The first thing we do is that
we make sure that there never will be placed a queen in a row if there's already a queen in that row. This is done by maintaining
a set with currently unused rows. \\
We also ensure that we never place a queen in a position which leads to a direct diagonal conflict. This is done by checking the
''slope'' between the position we are testing and all the queens already placed. If we look at the chess board as a grid we observe
that if any two queens have a line with slope 1 between them, they are on the same diagonal.
Two queens, named $A$ and $B$, are on the same diagonal if, and only if, one of the two following equalities hold:
\begin{align}
  row_A + col_A = row_B + col_B \\[1.3ex]
  col_A - row_A = col_B - row_B
\end{align}
Notice that these equations can be switched around in multiple ways. We chose this exact way since we think this is the least
computational expensive way of performing this check.

% Part 2
\section{Part 2}
\subsection{Solutions}
With 6 8 10 12 as starting positions we got the following results on a Intel Xeon 2.67GHz machine:
\begin{table}[ht]
  \centering
  \begin{tabu}{c | c | c}
     N & \# of solutions & Time used (seconds)\\ \hline
    14 &      77 &   0.01 \\
    16 &    1917 &   0.25 \\
    18 &   39804 &   7.98 \\
    20 & 1273580 & 367.24 
  \end{tabu}
  \caption{Results backtracking}
\end{table}	

% Part 3
\section{Part 3}
\subsection{Solutions}
\subsection{Tabu Search}
\subsection{Simulated Annealing}
\subsection{Genetic Algorithm}
\subsection{Comparison}
\begin{table}[ht]
  \centering
  \begin{tabu}{c | c | c | c | c | c | c }
      & \multicolumn{3}{c|}{\# of solutions} & \multicolumn{3}{c}{Time (seconds)}\\ \hline &&&&&&\\[-.3cm]
      N  &    TS &   SA & GA &     TS &     SA & GA \\ \hline &&&&&&\\[-.3cm]
     18  & 11120 & 3190 &    & 150.00 & 150.01 & \\ 
     20  &  8872 & 2312 &    & 150.00 & 150.01 & \\ 
     24  &  6125 & 1255 &    & 150.01 & 150.04 & \\ 
     30  &  3235 &  594 &    & 150.03 & 150.30 & \\ 
  \end{tabu}
  \caption{Results Tabu Search, Simulated Annealing, Genetic Algorithm}
\end{table}	

\end{document}
